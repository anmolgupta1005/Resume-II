% Awesome Source CV LaTeX Template
%
% This template has been downloaded from:
% https://github.com/darwiin/awesome-neue-latex-cv
%
% Author:
% Christophe Roger
%
% Template license:
% CC BY-SA 4.0 (https://creativecommons.org/licenses/by-sa/4.0/)

%Section: Project
\emptySeparator
\sectionTitle{Projects}{\faLaptop}
\begin{projects}
	\project
	{High Amplitude Audio Impulse Detection}{2019-Current}
	{}
	{Evaluate architectural optimizations to detect High Amplitude Audio Impulses in an indoor scenario. Point man for Analog Garage on evaluating advanced on-edge AI micro-processors, and ultra-low power sensory front-ends to enable battery operated, IoT devices; opening doors to a myriad of applications for ADI. Initial development of models and feature evaluation is deployed on Amazon Web Services. The project is currently in the Angel Phase of development}
	{Python3, Tensorflow, Scikit, AWS, Amazon SageMaker, AI-DNNs, MEMS-Mic}

	\project
	{Machine Health Monitoring}{2018}
	{\website{https://www.otosense.com}{Otosense}}
	{Implemented a processing pipeline that includes multi-interface sensing system, feature extraction, edge processing, and connection to back-end cloud service that in cumulation detects anomalies in assets using Audio and Vibration signature. Interfaces included M12, USBs, AUX LineIN, and POE}
	{C, Python3, Direct Acyclic Graph (DAG), Audio Processing, Machine Learning, Modular Architecture, Test-Driven Programming}

	\project
	{Assessing Malware Detection using Hardware Performance Counters (HPCs)}{2016-2017}
	{\website{https://hdl.handle.net/2144/27051}{MS-THESIS}}
	{Examined and proved that it is NOT possible to classify high-level behavior of a program (whether it is Malware or not) using the HPCs. Experiments included profiling more than 1000 Malware programs using the HPCs. Six different Machine Learning algorithms provided by the Scikit library tried to classify these programs as Malware. For sampling the HPCs, developed \textit{Savitor}, a tool that counts hardware events associated to a program. Using \textit{Savitor}, a self-developed tool, overcame the restrictions posed by AMD’s proprietary software CodeAnalyst, to provide flexibility and efficiency in data collection for our experiments.}
	{C++, Python3, Scikit, Machine Learning, Malware, Security Assessment}

	% \project
	% {Security Assessment of Bitcoin}{April'17}
	% {As a security assessment for the cybersecurity course, highlighted the various attack surfaces that are exploited by Wallet Vulnerabilities, Time Jacking and Transaction Malleability (vulnerability exploited in the famous \href{http://www.darkreading.com/attacks-and-breaches/mt-gox-bitcoin-meltdown-what-went-wrong/d/d-id/1114091}{\textit{Mt. Gox}} attack)}
	% {Bitcoin \faBitcoin, Cybersecurity}
				
	\project
	{MBTA-Live-Tracker}{April'16}
	{\github{anmolgupta1005/MBTA-Live-Tracker} \href{https://www.youtube.com/watch?v=DtY4qqCeVRI}{\faYoutube Video Link}}
	{Designed an embedded system with touch screen based GUI (on QT) for live tracking of Boston’s public transport system - the MBTA, using GUMSTIX and RASPBERRY PI controllers}
	{C, C++, Python2, Gumstix Verdex Pro, Raspberry Pi, QT }

	\project
	{Rush Hour}{December'15}
	{\github{anmolgupta1005/Rush-Hour} \href{https://youtu.be/s36H25OkzVQ}{\faYoutube Video Link}}
	{Interfaced a keyboard and a HDMI display monitor to Nexys-3 (based on Xilinx Spartan-6 LX16 FPGA) board to make a video game}
	{Verilog, HDL, Xilinx FPGA, Nexys Project, Digital Modeling}
\end{projects}

%\sectionTitle{Projets}{\faLaptop}
%\twocolumnsection{
%	\begin{projects}
%		\project
%			{YAAC Another Awesome CV}{2013 - 2018}
%			{\github{darwiin/yaac-another-awesome-cv} }
%			{Template \LaTeX pour la réalisation de Curiculum Vitæ qui utilise \href{https://fontawesome.com}{Font Awesome} et la police de caractère Adobe Source.}
%		{\LaTeX,Sublime Text}
%	\end{projects}
%}
%{\begin{projects}
%	\project
%	{Simply City}{2017 - 2018}
%	{\github{darwiin} \website{https://www.simplycity.nc}{https://www.simplycity.nc}}
%	{Igitur nam locis plane homines quidem et locis dicit quot quidem quod fallare si sed satisfacit intellegam et falli dicit mihi igitur possumus locis admodum eloquentiam vult et non admodum complectitur quod intellegam et intellegam complectitur et tamen philosopho quidem si vult igitur locis falli ego non philosophi habeat Torquate et non inquam pluribus asperner verbis dicit igitur sententiae locis non si sententiae oratio et non quot aeque habeat non homines habeat si inquam quod ego istius istius quidem ego pluribus aeque oratio non quidem pluribus verbis si igitur tamen nam istius vult non tot istius et ego non non.}
%	{DB2,Eclipse,Infosphere Traceability Server}
%	\end{projects}
%}